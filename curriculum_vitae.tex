\documentclass[10pt,a4paper]{altacv}
\usepackage{cv-style}
\begin{document}
    \name{Angela Cortecchia}
    \tagline{Ph.D. student}
    \personalinfo{%
        % You can add your own with \printinfo{symbol}{detail}
        \email{angela.cortecchia@unibo.it}
        \email{angela.cortecchia@hotmail.com}
    %   \phone{+49-173-6895039} 
        \\
        \github{github.com/angelacorte}
        \linkedin{linkedin.com/in/angela-cortecchia}
        \\
        \location{Cesena, Italy}

        %\orcid{orcid.org/0000-0000-0000-0000} % Obviously making this up too. If you want to use this field (and also other academicons symbols), add "academicons" option to \documentclass{altacv}
    }

    %% Make the header extend all the way to the right, if you want. Extend the right margin by 8cm (=6.8cm marginparwidth + 1.2cm marginparsep)
    \begin{adjustwidth}{}{-8cm}
        \makecvheader
    \end{adjustwidth}

\begin{adjustwidth}{}{-9cm}
    \cvsection{Education}
    \cvblock{Ph.D. Student in Computer Science}{University of Bologna}{11/2024 - ongoing}{Cesena, Italy}
    \textbf{Topic:} Towards Collective Operating Systems through Aggregate Computing\newline
    \textbf{Supervisor: Danilo Pianini} \newline \textbf{Co-Supervisor: Mirko Viroli}

    \divider

    \cvblock{M.Sc. in Engineering and Computer Science}{University of Bologna}{2021--2024}{Cesena, Italy}
    \textbf{109/110} \newline
    \textbf{Thesis}: A Kotlin Multiplatform implementation of Aggregate Computing based on XC \newline
    \textbf{Supervisor: Danilo Pianini} \newline \textbf{Co-Supervisor: Nicolas Farabegoli}

    \divider

    \cvblock{B.Sc. in Engineering and Computer Science}{University of Bologna}{2017--2021}{Cesena, Italy}
    \textbf{Thesis}: HCI Methodologies for Developing an Online Cultural Events App\newline
    \textbf{Supervisor: Silvia Mirri}

    \divider

    \cvblock{Undergraduate Internship}{University of Bologna}{10-12/2020}{Cesena, Italy}
    Developing an online cultural events app\newline
    \textbf{Supervisor: Silvia Mirri}

    \divider

    \cvblock{Accounting degree in Business and Foreign Language}{ITS "A. Oriani"}{2017}{Faenza, Italy}

    \divider 
\end{adjustwidth}

\begin{adjustwidth}{}{-9cm}
    \cvsection{Teaching Experience}

    \cvblock{Teaching Tutor}{University of Bologna}{01/2025 - 09/2025}{Cesena, Italy}
    Teaching tutor for the course ``Architetture degli Elaboratori'' in the Bachelor's Degree in Computer Science and Engineering.\newline

    \divider

    \cvblock{Teaching Tutor}{University of Bologna}{09/2025 - ongoing}{Cesena, Italy}
    Teaching tutor for the course ``Programmazione ad Oggetti'' in the Bachelor's Degree in Computer Science and Engineering.\newline

    \divider
\end{adjustwidth}

\begin{adjustwidth}{}{-9cm}
    \cvsection{Publications}

    \cvsubsection{\color{subheading}Published}
    \nocite{*}
    \printbibliography[heading=none, check=pub]
    \divider
    % \cvsubsection{\color{subheading}Accepted -- {\small Available soon}}
    % \printbibliography[heading=none, check=notpub]
    \cvsubsection{\color{subheading}Accepted -- {\small Available soon}}
    \printbibliography[heading=none, check=notpub]

    \divider

\end{adjustwidth}

\begin{adjustwidth}{}{-9cm}
    \cvsection{Research Contracts}
    \cvblock{Research Fellowship ``A Unifying Approach to Programming Heterogeneous Devices in the Edge-Cloud Continuum''}{``Group for Research Networks Harmonisation'' Consortium GARR at DISI - University of Bologna}{15/02/2024 -- 31/10/2024}{Cesena, Italy}
    \textbf{Supervisor: Danilo Pianini}

    \divider

\end{adjustwidth}

\begin{adjustwidth}{}{-10cm}
    \cvsection{Awards}

    \cvblock{Participation Grant}{ACSOS 2024 {\newline \scriptsize5th IEEE International Conference on Autonomic Computing and Self-Organizing Systems}}{16 Sep 2024}{Aarhus, Denmark}

    \divider

    \cvblock{Best Student Paper Award}{ACSOS 2025 {\newline \scriptsize5th IEEE International Conference on Autonomic Computing and Self-Organizing Systems}
    \newline Paper: ``A Field-based Approach for Runtime Replanning in Swarm Robotics Missions''}{29 Sep 2025}{Tokyo, Japan}

    \divider
\end{adjustwidth}

\begin{adjustwidth}{}{-10cm}
    \cvsection{Attended Challenges}

    \cvblock{Flash Talk Challenge}{{RFTC 2024 \newline \scriptsize1st Researchers Flash-Talk Challenge}}{11 Oct 2024}{Bologna, Italy}

   \divider

\end{adjustwidth}

\begin{adjustwidth}{}{-10cm}
    \cvsection{Volunteer Experiences}

    \cvblock{DS-RT Local Chair Helper}{{DS-RT 2024 \newline \scriptsize28th International Symposium on Distributed Simulation and Real Time
    Applications}}{7-9 Oct 2024}{Urbino, Italy}

   \divider

\end{adjustwidth}

\begin{adjustwidth}{}{-10cm}
    \cvsection{Attended Conferences}

    \cvblock{Presented at the Main Track and PhD Symposium Track}{{ACSOS 2024 \newline \scriptsize5th IEEE International Conference on Autonomic Computing and Self-Organizing Systems}}
    {16-20 Sep 2024}{Aarhus, Denmark}

    \divider

    \cvblock{Presented at the PhD Symposium Track}{{DS-RT 2024 \newline \scriptsize28th International Symposium on Distributed Simulation and Real Time
    Applications}}{7-9 Oct 2024}{Urbino, Italy}

   \divider

\end{adjustwidth}

\begin{adjustwidth}{}{-10cm}
    \cvsection{Attended Summer Schools} 

    \cvblock{}{{SPACERAISE 2025 \newline \scriptsize The ``International Doctoral School'' for the Space Sector}}
    {12-16 May 2025}{L'Aquila, Italy}

    \divider

    \cvblock{}{{BISS 2025 \newline \scriptsize The ``Bertinoro International Spring School on Software Engineering''}}
    {19-23 May 2025}{Bertinoro, Italy}

    \divider

    \cvblock{}{{SIESTA 2025 \newline \scriptsize 5th International Software Engineering Summer School}}
    {27-29 August 2025}{Lugano, Switzerland}

    \divider

\end{adjustwidth}

\begin{adjustwidth}{}{-9cm}
    \cvsection{Previous work experience}

    \cvblock{Horse Rider and Coach}{Delta Team}{2020--2024}{Alfonsine, Italy}
    Practicing horse riding at an international competitive level for several years, training and competing young horses. \smallbreak
    %operatore ludico
    Done the coaching level ``Operatore Ludico'' of the Italian Equestrian Federation (``Federazione Italiana Sport Equestri'' - FISE) coaching course in 2021. \smallbreak

    \divider
    
\end{adjustwidth}

\begin{adjustwidth}{}{-9cm}
    \cvsection{Professional Skills}

    \cvskillblock{Programming Languages}{Kotlin, Java, Scala, Javascript, Typescript, C, Bash, Python, PHP, Sql, Prolog}
    \divider

    \cvskillblock{Other Languages}{Markdown, YAML, \LaTeX{}, HTML, CSS, XML, JSON}
    \divider

    \cvskillblock{Technologies}{Git, Docker, Angular, Android, Vue, Svelte, MEAN}
    \divider

    \cvskillblock{Programming Paradigms}{Object Oriented Programming, Functional Programming, Aggregate Programming, Logic Programming, Event Driven Programming, Concurrent \& Distributed Programming}
    \divider

    % \cvskillblock{Project Management}{Scrum, Agile}
    % \divider
\end{adjustwidth}

%\clearpage

\begin{adjustwidth}{}{-9cm}
    \cvsection{Portfolio}

    \textbf{Collektive}\\
    \github{\href{https://github.com/Collektive}{github.com/Collektive}}\\
    \textit{10/2023--ongoing}\\ \smallskip
    Collektive: A Kotlin Multiplatform implementation of Aggregate Computing based on XC. The project was extended for academic purposes for the Master Degree Thesis.\\ \smallskip
%    {\small \notesymbol \hspace{0.5em} Paper published to EUMAS 2023}\\
    \smallskip
    \textbf{Language:} Kotlin\\
    \textbf{Keywords:} Domain Specific Language, Aggregate Computing, Field Calculus, eXchange Calculus

%    \divider
%
%    \textbf{FieldVMC}\\
%    \github{\href{https://github.com/angelacorte/fieldVMC}{github.com/angelacorte/fieldVMC}}\\
%    \textit{03--05/2024}\\ \smallskip
%    This artifact was born for the evaluation of \emph{FieldVMC}: a generalisation of the VMC model as a field-based computation,
%    in the spirit of the Aggregate Programming (AP) paradigm.
%    Work related to the paper ``An Aggregate Vascular Morphogenesis Controller for Engineered Self-Organising Spatial Structures''.\\ \smallskip
%    {\small \notesymbol \hspace{0.5em} Paper published to ACSOS 2024}\\
%    \smallskip
%    \textbf{Language:} Kotlin\\
%    \textbf{Keywords:} Aggregate Computing, Self-Organizing, Vascular Morphogenesis Controller

    \divider

    \textbf{VMC-experiments}\\
    \github{\href{https://github.com/angelacorte/vmc-experiments}{github.com/angelacorte/vmc-experiments}}\\
    \textit{03--05/2024}\\ \smallskip
    This artifact was born for the evaluation of \emph{FieldVMC}: a generalisation of the VMC model as a field-based computation, 
    in the spirit of the Aggregate Programming (AP) paradigm.
    Work related to the paper ``An Aggregate Vascular Morphogenesis Controller for Engineered Self-Organising Spatial Structures''.\\ \smallskip
    {\small \notesymbol \hspace{0.5em} Paper published to ACSOS 2024}\\
    \smallskip
    \textbf{Language:} Kotlin\\
    \textbf{Keywords:} Aggregate Computing, Self-Organizing, Vascular Morphogenesis Controller

    \divider

    \textbf{Collektive-examples}\\
    \github{\href{https://github.com/Collektive/collektive-examples}{github.com/Collektive/collektive-examples}}\\
    \textit{01--03/2024}\\ \smallskip
    Examples of the Collektive project. The examples were made for academic purposes for the Master Degree Thesis and Research Grant.\\ \smallskip
    \smallskip
    \textbf{Language:} Kotlin\\
    \textbf{Keywords:} Domain Specific Language, Aggregate Computing, Field Calculus, eXchange Calculus

    \divider

    \textbf{Rust Fields}\\
    \github{\href{https://github.com/RustFields}{github.com/RustFields}}\\
    \textit{05--09/2023}\\ \smallskip
    This project was born from the willing to extend ScaFi: a Scala-based library and framework for Aggregate Programming.
    The goal of this project is to explore different solutions to make the field calculus available on thin devices.
    The topic was used for the development of two projects for three different Master Degree courses: ``Pervasive Computing'',
    ``Laboratory of Software Systems'', and ``Project Management''.\\ \smallskip
    \textbf{Language:} Rust, Scala\\
    \textbf{Keywords:} Aggregate Computing, Field Calculus, Rust, ScaFi, Pervasive Computing

    \divider

    \textbf{Equilessons}\\
    \github{\href{https://github.com/angelacorte/equilessons}{https://github.com/angelacorte/equilessons}}\\
    \textit{2021/2023}\\ \smallskip
    The aim of the project was to optimize the management of lessons of riding schools.
    The project was made for leisure purposes before the start of the Master Degree, it was then extended for academic purposes for the course ``Applicazioni e Servizi Web'' in the Master Degree.\\ \smallskip
    \smallskip

    \textbf{Language:} Typescript\\
    \textbf{Keywords:} Web Application, Angular, MEAN, Typescript

    \divider

    \textbf{Sette e Mezzo clone}\\
    \github{\href{https://github.com/angelacorte/SetteMezzo-Clone}{github.com/angelacorte/SetteMezzo-Clone}}\\
    \textit{08--10/2022}\\ \smallskip
    A digital and distributed command-line application of the Italian card game Sette e Mezzo. The project was made for academic purposes under the course ``Distributed Systems'' in the Master Degree.\\ \smallskip
    \textbf{Language:} Typescript\\
    \textbf{Keywords:} CLI application, Distributed Systems, Typescript

    \divider

    \textbf{Smart Charging Stations}\\
    \github{\href{https://github.com/angelacorte/smart-charging-station-report}{github.com/angelacorte/smart-charging-station-report}}\\
    \textit{07--09/2023} \\ \smallskip
    Web application for the management of charging stations for electric vehicles. The project was made for academic purposes under the course ``Smart City e Tecnologie Mobili'' in the Master Degree.\\ \smallskip
    \textbf{Language:} Scala, Typescript, Svelte, NodeJS\\
    \textbf{Keywords:} Web Application, Smart City, Scala, Typescript, Svelte, Akka Actors

    \divider

    \textbf{PPS-galaxy sim}\\
    \github{\href{https://github.com/FilippoVissani/PPS-22-galaxy-sim}{github.com/FilippoVissani/PPS-22-galaxy-sim}}\\
    \textit{07--09/2023} \\ \smallskip
    %è un simulatore del moto dei corpi all’interno di una galassia.
    A simulator of the motion of bodies within a galaxy. The project was made for academic purposes under the course ``Paradigmi di Programmazione e Sviluppo'' in the Master Degree.\\ \smallskip
    \textbf{Language:} Scala
    \textbf{Keywords:} Simulation, Scala, Akka Actors

\end{adjustwidth}

\begin{adjustwidth}{}{-9cm}
    \cvsection{INTERESTS}

    \cvtag{Aggregate Computing}
    \cvtag{CI \& CD}
    \cvtag{Swarms}
    \cvtag{User Experience and User Interface}
    \cvtag{Web and Mobile Development}
    \cvtag{Horse Riding}
    \cvtag{Astronomy}

    \divider
\end{adjustwidth}



\end{document}
